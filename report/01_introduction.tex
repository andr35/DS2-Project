\section{Introduction}
\label{sec:introduction}

A failure detector can be described as a distributed oracle that provides information about the state of the other processes in the system.
From a practical point of view, it is a service that detects when the other processes are not available, for instance when they crash or become not reachable.
Failure detectors can be used to solve difficult problems in distributed systems, such as reliable broadcast, consensus and group membership.

A simple way to build a failure detector is to periodically ping each other process.
On the one hand, the idea is effective and easy to implement; on the other hand, the resulting protocol is very inefficient, since it produces $O(n^2)$ messages every ``round'' (where $n$ is the number of the processes).
For extreme distributed systems, this approach does not scale.

An alternative approach to build failure detectors is to use gossip.
In gossip-based protocols, processes exchange information with random peers at regular intervals.
In our case, they exchange information about the processes which are still gossiping:
if a process does not gossip anymore, it is probably crashed or unreachable.
This approach was firstly proposed by Van Renesse, Minsky and Hayden in 1998 in the paper ``A Gossip-Style Failure Detection Service'' \cite{gsfd}.
The authors describe a protocol based on gossip to provide timely and efficient failure detection and provide a backup protocol based on broadcast to recover from catastrophes (for example, when half of the nodes crashes at the same time).

In this work, we implement and test a protocol based on the original paper.
We try some modifications to the original proposal and measure their effect.
We also tune the performances to match the computing power and network capabilities of modern distributed systems.

\cref{sec:paper} describes in detail the original work.
\cref{sec:variants} shows some little modifications to improve the protocol.
\cref{sec:implementation} describes how we implement the protocol.
\cref{sec:metrics} shows how we evaluate and test the protocol.
Finally, \cref{sec:experiments} discuss which modifications help the original protocol and which do not.
To conclude, we describe a simple yet effective way to tune the performances based on real world needs.
