\section{Conclusions}
\label{sec:conclusions}

In this work, we implemented a gossip-based protocol based on the paper ``A Gossip-Style Failure Detection Service'' of Van Renesse, Minsky and Hayden.
We tested and validated with multiple experiments in normal and extreme conditions the validity of the gossip approach to build a perfect failure detector.
We also proposed some variants to the protocol, such as different strategies for the peer selection and the Push-Pull approach for the gossip protocol.

The results of our experiments confirm the power of gossip protocols to solve difficult problems in an efficient way.
The Push-Pull strategy achieved better performances in all conditions.
Even though the analytical proof is very complex and out of the scope of this work, we strongly believe the Push-Pull strategy to outperform Push in almost every scenario.

In general, the alternative strategies proposed to select gossip recipients achieve performances comparable to the original one (pure random choice).
In the case of catastrophes, they are detrimental.
The randomness of the original strategy helps the protocol to better survive in extreme conditions.
Although is it probably possible to find out better strategies under very specific hypothesis, we believe it is not worth.
The random choice resulted to be very simple to implement, resilient and efficient.

The backup multicast algorithm is fundamental to recover from catastrophic scenarios such as the sudden and synchronized crash of \nicefrac{2}{3} of the nodes in the network.
This may be realistic in very extreme cases, but very unlikely in real world applications.
Depending on the specific scenario, one can choose to turn on or off the backup protocol.

The Internet has changed a lot from 1998, when the original work was published.
With the modern infrastructure, network capabilities and computing power, we do not need the level of optimization of resources consumptions needed in 1998.
This allows us to arbitrary tune the performances wanted from such a protocol depending on our needs.
In particular, a very simple way to do this is to change the gossip time.
Lower gossip times will produce a higher load on the infrastructure, but will achieve very fast failure detections.
Since the average failure detection time scales as \nobreak{$O(n\nobreak\hspace{.08em}\cdot\nobreak\hspace{.08em}log(n))$} with respect to the number of nodes, we can achieve very good results with a moderate consumption of network resources.

